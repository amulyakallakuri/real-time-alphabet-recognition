\documentclass{article}

% If you're new to LaTeX, here's some short tutorials:
% https://www.overleaf.com/learn/latex/Learn_LaTeX_in_30_minutes
% https://en.wikibooks.org/wiki/LaTeX/Basics

% Formatting
\usepackage[utf8]{inputenc}
\usepackage[margin=1in]{geometry}
\usepackage[titletoc,title]{appendix}

% Math
% https://www.overleaf.com/learn/latex/Mathematical_expressions
% https://en.wikibooks.org/wiki/LaTeX/Mathematics
\usepackage{amsmath,amsfonts,amssymb,mathtools}

% Images
% https://www.overleaf.com/learn/latex/Inserting_Images
% https://en.wikibooks.org/wiki/LaTeX/Floats,_Figures_and_Captions
\usepackage{graphicx,float}

% Tables
% https://www.overleaf.com/learn/latex/Tables
% https://en.wikibooks.org/wiki/LaTeX/Tables

% Algorithms
% https://www.overleaf.com/learn/latex/algorithms
% https://en.wikibooks.org/wiki/LaTeX/Algorithms
\usepackage[ruled,vlined]{algorithm2e}
\usepackage{algorithmic}

\usepackage{hyperref}
\hypersetup{
    colorlinks=true,
    linkcolor=blue,
    filecolor=magenta,      
    urlcolor=cyan,
    pdftitle={Overleaf Example},
    pdfpagemode=FullScreen,
    }

\urlstyle{same}


% Title content
\title{EE541 Final Project Proposal}
\author{Amulya Kallakuri, Hongbo Zhang}
\date{April 12, 2022}

\begin{document}

\maketitle


\section{Project Title:} Real-time Alphabet Recognition for American Sign Language Alphabet

\section{Topic summary:} American Sign Language (ASL) is a natural language that serves as the primary mode of communication for deaf people in the United States of America. 
 ASL is a visual language, in that it is between a gesturer and an observer with the use of hand 
 movements. There are five components in the identification of an ASL symbol: handshape, palm 
 orientation, movement, location, and expression/non-manual signals, each of which impacts the meaning of the gesture. \\ \\ Communicating with ASL is a challenge for people that are not deaf 
 and have limited or zero knowledge of the language. It is of paramount importance that people 
 of essential services, for instance, are able to accurately convey information to deaf people and vice versa. 
 This project is our attempt to help and alleviate the discomforts faced by deaf people and to 
 enable them to get one step closer to eliminating the boundary between people that can speak and 
 people that cannot. \\ \\ There has been considerable research done in the field of using Deep Learning 
 to convert ASL to text or speech, all procuring varying results. We propose to use our knowledge 
 and literary review of a significant number of these papers to create a model to detect and 
 transcribe American Sign Language in real-time and to alert the user when a gesture is not part of the ASL. 
 Additionally, we shall also aim to transcribe this text into speech and autocorrect the indentified words, if time permits. 

\section{Dataset description:} The original training dataset is 1.27 GB 
containing 29 classes with 3000 images, each of which is a 200x200 RGB image. 
We plan on augmenting this data and creating a testing dataset with images of 
our own, varying in background, skin colour. This generated data shall account for all
five factors in the identification of ASL and will try to reasonably accommodate for them. We would also augment the training 
data with signs that are incorrect in ASL to train the model to identify when the 
sign shown is not part of the ASL.

\section{Architecture Investigation Plan:} Generally speaking, the static ASL alphabet is an easy classification task in computer vision.
According to previous papers and codebases, deep convolution models will produce a satisfying accuracy.
We will investigate several CNN backbones, including VGG-16, ResNet50, and EfficientNetV2L, and then compare their Top-1 and Top-5 accuracy. Since we plan to build a real-time application, we'll also compare the inference FPS of each model on our experimental platform.
The final architecture will be balanced between accuracy and speed.\\ \\

\section{Estimated Compute Needs:} We shall use a personal workstation 
with NVIDIA 3080 Ti, AMD 5800x and 32GB RAM. We shall also consider using online GPU acceleration from Google Colaboratory or Kaggle if neccessary.

\section{Primary References and Codebase:}
1. Sharma, Shikhar, and Krishan Kumar. "ASL-3DCNN: American sign language recognition technique using 3-D convolutional neural networks." Multimedia Tools and Applications 80.17 (2021): 26319-26331.\\ \\
2. Starner, Thad, Joshua Weaver, and Alex Pentland. "Real-time american sign language recognition using desk and wearable computer based video." IEEE Transactions on pattern analysis and machine intelligence 20.12 (1998): 1371-1375.\\ \\
3. Kadhim, Rasha Amer, and Muntadher Khamees. "A real-time american sign language recognition system using convolutional neural network for real datasets." Tem Journal 9.3 (2020): 937.\\ \\
4. Simonyan, Karen, and Andrew Zisserman. "Very deep convolutional networks for large-scale image recognition." arXiv preprint arXiv:1409.1556 (2014).\\ \\
5. He, Kaiming, et al. "Deep residual learning for image recognition." Proceedings of the IEEE conference on computer vision and pattern recognition. 2016.\\ \\
6. Tan, Mingxing, and Quoc Le. "Efficientnetv2: Smaller models and faster training." International Conference on Machine Learning. PMLR, 2021.\\ \\
Github codebases: Mohamed Y.Helmy, \href{https://github.com/Mohamedyasserhelmy/Sign-Language-Translator-ASL}{Sign Language Translator} \\ \\
Dataset: Akash, \href{https://www.kaggle.com/datasets/grassknoted/asl-alphabet/code}{ASL Alphabet}, 
Image data set for alphabets in the American Sign Language

\end{document}